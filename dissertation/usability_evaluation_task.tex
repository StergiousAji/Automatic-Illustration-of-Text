\documentclass[]{article}

\usepackage{amsmath,amssymb,graphicx, fancyvrb}
 
\title{\textbf{Automatic Illustration of Text}\\ \vspace{6mm}
    \Large Text-to-Image Relevance Assessment Task}

\date{}

\pagenumbering{gobble}

\begin{document}
\maketitle
\noindent
This document explains the experiment that you are about take part in. You will be asked to use a prototype application that provides the capability to construct ground truths for automatic videography generation tasks. The experiment aims to measure the usability of the interface and so you will not be given detailed instructions as to how to complete the task. You must use what is available on the web pages to navigate through the application and complete your objectives.

\vspace{6mm}
\noindent
The task will involve constructing a ground truth using the application for the provided audio source below. You are free to browse YouTube or the local device for the audio source. Once the appropriate source is found, you should input this into the application and can subsequently start annotating the true labels for each textual chunk within the audio. You will be given 10 minutes but you are not expected to finish annotating the full ground truth.

\vspace{6mm}
\noindent
Your chosen audio source is: 

\begin{center}
    \textbf{[AUDIO SOURCE INFORMATION]}
\end{center}

\vspace{6mm}
\noindent
Once, the experiment has concluded, you will be asked to complete a short usability survey of your experience interacting with the web application. The data gathered will be anonymised and no personal details will be collected.

\end{document}